\section{Key Takeaways}

\begin{itemize}
	\item \textbf{Real $\theta (t)$ vs. Eq. (3):} Comparing our experimental data with Eq. (3) revealed discrepancies attributable to non-idealities. While the overall trend might resemble the cosine function, air resistance and other dissipative forces cause the amplitude to gradually decrease over time.
	\item \textbf{Mechanical Energy Conservation:} Our calculations confirmed that \hl{mechanical energy of the pendulum} is not perfectly conserved in our set-up. We observed a progressive decline in $E$ from $135kJ$ at $t_{1}$ to $71kJ$ at $t_{7}$, representing a total loss of approximately $47\%$. This decrease stems from the dissipative forces that extract energy from the system and ultimately dampen the oscillations.
	\item \textbf{Beyond the Ideal Model:} Recognizing the limitations of the ideal model is crucial for understanding real-world pendulum dynamics. By accounting for air resistance, joint friction, and other non-idealities, we can refine our theoretical models and develop more accurate predictions for practical applications.
\end{itemize}
