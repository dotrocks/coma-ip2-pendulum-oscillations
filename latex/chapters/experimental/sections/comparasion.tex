\section{Comparing Real $\theta (t)$ to Eq. (3)}

While the simple harmonic motion model described by Eq. (3) provides a valuable framework for understanding pendulum oscillations, it's crucial to recognize that real-world pendulums deviate from this idealized scenario due to various factors. Here, we explore the main reasons why the actual behaviour of a pendulum might differ from the theoretical prediction:

\begin{itemize}
	\item \textbf{Air Resistance:} As the pendulum swings, it encounters friction with the surrounding air. This dissipative force acts against the motion, gradually decreasing the amplitude of the oscillations over time. Unlike the ideal model where energy remains constant, air resistance causes the mechanical energy to be lost to heat, resulting in a gradual decay of the oscillations.
	\item \textbf{Joint Resistance:} No pivot point is perfectly frictionless. In a real pendulum, the joint connecting the string to the support point introduces a frictional force that opposes the motion. This frictional dissipation also contributes to the loss of energy and the eventual damping of the oscillations. The magnitude of this effect depends on the materials and construction of the pivot point.
	\item \textbf{Non-ideal String Mass:} While the model often assumes a massless string, real pendulums use strings with a finite mass. This additional mass affects the overall dynamics of the system, slightly modifying the period and energy distribution compared to the theoretical case. The impact of string mass becomes more significant as its weight approaches a non-negligible fraction of the bob's mass.
	\item \textbf{Large Angular Displacements:} Eq. (3) relies on the small-angle approximation, assuming $\sin(\theta) \approx \theta$. However, for large angular excursions, this approximation breaks down, leading to deviations from the ideal cosine function behaviour. In real pendulums, especially those with large maximum deviations, the true oscillation pattern might exhibit slight distortions compared to the predicted smooth cosine wave. 
	
	\newpage
	\thispagestyle{plain}

	\item \textbf{Non-linear Restoring Forces:} The model assumes a linear restoring force proportional to the angle of deviation. However, in some cases, the actual restoring force might be slightly non-linear, particularly at larger angles. This non-linearity can introduce higher-order harmonic terms into the motion, causing the pendulum to deviate from the simple harmonic behaviour predicted by Eq. (3).
\end{itemize}

By acknowledging these real-world limitations, we gain a more nuanced understanding of pendulum dynamics. These departures from the ideal model highlight the importance of considering environmental factors and non-idealities when interpreting the behaviour of real-world systems. By accounting for these complexities, we can refine our theoretical models and develop more accurate predictions for practical applications.
