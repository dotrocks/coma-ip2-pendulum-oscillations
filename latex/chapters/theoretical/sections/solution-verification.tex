\section{Solution Verification}

\begin{itemize}
	\item Substitute the function into Eq. (2):
		\begin{itemize}
			\item Take the second derivative of $\theta = B\cos(\omega t)$ with respect to $t$: 
				\[\frac{\partial ^{2}\theta (t)}{\partial t^{2}} = -B\omega^{2}\cos(\omega t)\]
			\item Substitute $\theta (t) = B\cos(\omega t)$ and its second derivative into Eq. (2): 
				\[-B\omega^{2}\cos(\omega t) + \omega^{2}B\cos(\omega t) = 0\]
		\end{itemize}
	\item Simplify the equation:
		\begin{itemize}
			\item Notice that all terms cancel out, leaving $0 = 0$
		\end{itemize}
	\item Interpretation:
		\begin{itemize}
			\item Since the equation becomes true after substitution, it means that ${\theta (t) = B\cos(\omega t)}$ indeed satisfies Eq. (2)
		\end{itemize}
\end{itemize}

Therefore, we can conclude that the function $\theta (t) = B\cos(\omega t)$ is a valid solution to the equation of motion for the simple pendulum with a linear restoring force, as described in this scenario.

Remember that $B$ represents the amplitude of the oscillation and depends on the initial conditions of the pendulum. Additionally, $\omega$, the angular frequency, is related to the period $T$ by the equation  $\omega = \frac{1}{T}$
